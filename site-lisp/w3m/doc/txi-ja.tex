%% TeX macros to handle Japanese texinfo files
%%
%% 92.7.8   modified for Mule Ver.0.9.5 by K.Handa <handa@etl.go.jp>
%%      To detect type of jTeX and its version, the method
%%      posted by Takafumi SAKURAI <sakurai@math.metro-u.ac.jp> is used.
%% 92.9.30  modified for Mule Ver.0.9.6 by K.Handa <handa@etl.go.jp>
%%      For unknown reason, \newif\ifNTTOLD should be before
%%      \ifNTT.
%% 93.4.29  modified for Mule Ver.0.9.7 by N.Hikichi <hikichi@sra.co.jp>
%% 95.10.6  modified for texinfo 2.145 by K.Handa <handa@etl.go.jp>
%% 95.10.13 modified by J.Sato <jun@svgw.rd.casio.co.jp>
%%      Support many Japanese oriented phrases (reference, etc)
%% 95.10.14 modified by K.Handa <handa@etl.go.jp>
%%      Bug for handling index fixed.
%% 96.1.16 modified by J.Sato <jun@svgw.rd.casio.co.jp>
%%      index with [] of @deffn.
%% 99.6.27 modified by Moimoi <fukusaka@xa2.so-net.ne.jp>
%%	for texinfo 1999-05-25.6
%% 2000.2.23 modified by Moimoi <fukusaka@xa2.so-net.ne.jp>
%%	for texinfo.tex 1999-09-25.10
%% 2000.4.11 modified by Moimoi <fukusaka@xa2.so-net.ne.jp>
%%      for texinfo.tex 1999-09-25.10
%%      fixed for jTeX/pTeX/MulTeX
%% 2010.06.02 modified by Shoichi Fukusaka <fukusaka@xa2.so-net.ne.jp>
%%      for texinfo.tex 2005-01-30.17
%%      omitted support for old NTT-jTeX/MulTex
%%      improve macros
%% 2010.06.03 modified by Shoichi Fukusaka <fukusaka@xa2.so-net.ne.jp>
%%      for texinfo.tex 2010-05.22.17 modified version
%%      minimize re-define macro
%% 2010.06.03a modified by Shoichi Fukusaka <fukusaka@xa2.so-net.ne.jp>
%%      for texinfo.tex 2010-05.22.17 modified version
%%      output \special for dvipdfmx to generate pdf bookmark
%% 2015.10.13 modified by Katsumi Yamaoka <yamaoka@jpl.org>
%%      make the pdf document display a bookmark by default
%%%%%%%%%%%%%%%%%%%%%%%%%%%%%%%%%%%%%%%%%%%%%%%%%%%%%%%%%%%%%%%%
%% 92.7.8 by K.Handa
\newif\ifNTT
\ifx\gtfam\undefined
\NTTtrue
\else
\NTTfalse
\fi

%% TeX macros to handle Japanese texinfo files
%% 92/05/24 merged jtexinfo.tex (by H. Isozaki and N. Hikichi) into this
%% Created by Satoru Tomura (tomura@etl.go.jp)

\def\texinfoJPversion{5}

\ifNTT
\message{txi-ja (NTT JTeX) package [Version \texinfoJPversion]:}
\else
\message{txi-ja (ASCII pTeX) package [Version \texinfoJPversion]:}
\fi
\message{}

%%%%%%%%%%%%%%%%%%%%%%%%%%%%%%%%%%%%%%%%%%%%%%%%%%%%%%%%%%%%%%%%
% Set up fixed words for Japanese.
\gdef\putwordAppendix{付録}
\gdef\putwordChapter{章}
\gdef\putwordfile{ファイル}
\gdef\putwordin{in}
\gdef\putwordIndexIsEmpty{(索引が空です)}
\gdef\putwordIndexNonexistent{(索引がありません)}
\gdef\putwordInfo{Info}
\gdef\putwordInstanceVariableof{Instance Variable of}
\gdef\putwordMethodon{Method on}
\gdef\putwordNoTitle{無題}
\gdef\putwordof{of}
\gdef\putwordon{on}
\gdef\putwordpage{p.\gobble}
\gdef\putwordsection{節}
\gdef\putwordSection{節}
\gdef\putwordsee{を参照}
\gdef\putwordSee{を参照してください}
\gdef\putwordShortTOC{簡略目次}
\gdef\putwordTOC{目次}
%
\gdef\putwordMJan{1月}
\gdef\putwordMFeb{2月}
\gdef\putwordMMar{3月}
\gdef\putwordMApr{4月}
\gdef\putwordMMay{5月}
\gdef\putwordMJun{6月}
\gdef\putwordMJul{7月}
\gdef\putwordMAug{8月}
\gdef\putwordMSep{9月}
\gdef\putwordMOct{10月}
\gdef\putwordMNov{11月}
\gdef\putwordMDec{12月}
%
\gdef\putwordDefmac{マクロ}
\gdef\putwordDefspec{特殊フォーム}
\gdef\putwordDefvar{変数}
\gdef\putwordDefopt{オプション}
\gdef\putwordDeffunc{関数}

\def\today{\number\year 年 \number\month 月 \number\day 日}

%%%%%%%%%%%%%%%%%%%%%%%%%%%%%%%%%%%%%%%%%%%%%%%%%%%%%%%%%%%%%%%%
% 日本語組の段落インデントのデフォルトを設定
\global\defaultparindent = 1em
\global\def\suppressfirstparagraphindent{\relax}

% 日本人好きのギッシリ詰まった紙
\global\def\afourpaperj{{\globaldefs = 1
  \parskip = 3pt plus 2pt minus 1pt
  \textleading = 12pt
  %
  \internalpagesizes{248mm}{170mm}%
                    {\voffset}{-6mm}%
                    {\bindingoffset}{8mm}%
                    {297mm}{210mm}%
  %
  \tolerance = 750
  \hfuzz = 1pt
  \contentsrightmargin = 0pt
  \defbodyindent = 5mm
}}

%%%%%%%%%%%%%%%%%%%%%%%%%%%%%%%%%%%%%%%%%%%%%%%%%%%%%%%%%%%%%%%%
%% Utility routines.

\def\csdef#1{\expandafter\def\csname#1\endcsname}

\gdef\addcsbefore#1#2{%
  \cslet{orig#1}{#1}%
  \csdef{addcs#1}{#2}%
  \csdef{#1}{%
    \cslet{addcs}{addcs#1}%
    \futurelet\orig\addcsbeforeyyy\expandafter\empty\csname orig#1\endcsname
  }
}
\gdef\addcsbeforeyyy{\addcs\orig}

\gdef\addcsafter#1#2{%
  \cslet{orig#1}{#1}%
  \csdef{addcs#1}{#2}%
  \csdef{#1}{%
    \cslet{addcs}{addcs#1}%
    \futurelet\orig\addcsafteryyy\expandafter\empty\csname orig#1\endcsname
  }
}
\gdef\addcsafteryyy{\orig\addcs}

%%%%%%%%%%%%%%%%%%%%%%%%%%%%%%%%%%%%%%%%%%%%%%%%%%%%%%%%%%%%%%%%
%% 日本語フォント

\def\jafont{%
\ifNTT%
\let\next=\jTeXjafont%
\else%
\let\next=\pTeXjafont%
\fi%
\next}

\def\jTeXjafont#1#2#3#4{%
\def\tempa{#2}
\def\tempb{mc}
\ifx\tempa\tempb% mc
\expandafter\gjfont\csname#1\endcsname=dm#3 scaled #4%
\else% gt
\expandafter\gjfont\csname#1\endcsname=dg#3 scaled #4%
\fi%
}

\def\pTeXjafont#1#2#3#4{%
\def\tempa{#2}
\def\tempb{mc}
\ifx\tempa\tempb% mc
\global\expandafter\font\csname#1\endcsname=min#3 scaled #4%
\else% gt
\global\expandafter\font\csname#1\endcsname=goth#3 scaled #4%
\fi
}

%
% 日本語フォントの定義
%

% Text fonts (11.2pt, magstep1).
\jafont{textmc}{mc}{10}{\mainmagstep}
\jafont{textgt}{gt}{10}{\mainmagstep}

% Fonts for indices, footnotes, small examples (9pt).
\jafont{smallmc}{mc}{9}{1000}
\jafont{smallgt}{gt}{9}{1000}

% Fonts for small examples (8pt).
\jafont{smallermc}{mc}{8}{1000}
\jafont{smallergt}{gt}{8}{1000}

% Fonts for title page (20.4pt):
\jafont{titlemc}{mc}{10}{\magstep4}

% Chapter (and unnumbered) fonts (17.28pt).
\jafont{chapmc}{mc}{10}{\magstep3}
\jafont{chapgt}{gt}{10}{\magstep3}

% Section fonts (14.4pt).
\jafont{secmc}{mc}{10}{\magstep2}
\jafont{secgt}{gt}{10}{\magstep2}

% Subsection fonts (13.15pt).
\jafont{ssecmc}{mc}{10}{1315}
\jafont{ssecgt}{gt}{10}{1315}

% Reduced fonts for @acro in text (10pt).
\jafont{reducedmc}{mc}{10}{1000}
\jafont{reducedgt}{gt}{10}{1000}

% Fonts for short table of contents. (12pt)
\jafont{shortcontmc}{mc}{10}{\magstep1}
\jafont{shortcontgt}{gt}{10}{\magstep1}

% 95.11.2 by K.Handa
% Reduce Overfull/Underfull \hbox by relaxing these glues.
\ifNTT
\global\jintercharskip=0pt plus 0.5pt minus -0.2pt
\global\jasciikanjiskip=2.28854pt plus 0.5pt minus -0.2pt
\fi

%%%%%%%%%%%%%%%%%%%%%%%%%%%%%%%%%

\def\syncjfont#1#2{\addcsafter{#1}{\csname #2\endcsname}}
\def\addjfonts#1#2{\addcsafter{#1fonts}{\cslet{tenmc}{#2mc}\cslet{tengt}{#2gt}}}

%%%%%%%%%%%%%%%%%%%%%%%%%%%%%%%%%
\globaldefs = 1

\syncjfont{rm}{tenmc}
\syncjfont{bf}{tengt}
\syncjfont{sl}{tengt}
\syncjfont{authorrm}{secmc}
\syncjfont{authortt}{secgt}

\addjfonts{text}{text}
\addjfonts{title}{title}
\addjfonts{chap}{chap}
\addjfonts{sec}{sec}
\addjfonts{subsec}{ssec}
\addjfonts{small}{small}
\addjfonts{smaller}{smaller}
\addjfonts{reduced}{reduced}

\let\subsubsecfonts = \subsecfonts
\let\smallexamplefonts = \smallfonts

\textfonts \rm

\globaldefs = 0

%%%%%%%%%%%%%%%%%%%%%%%%%%%%%%%%%%%%%%%%%%%%%%%%%%%%%%%%%%%%%%%%%%
% 日本語表記の改善
%  1. ヘッドの章の表記
%  2. See 訳語に合わせて動詞の後置
%  3. クロスレファレンスの表記
%  4. @dfn の表記

%%%%%%%%%%%%%%%%%%%%%%%%%%%%%%%%%
%  1. ヘッドの章の表記
\gdef\thischapterspace{\hskip \SETthischapterspace em}
\set thischapterspace 1

\gdef\printheadingappendix#1#2{\putwordAppendix{}#1\thischapterspace#2}
\gdef\printheadingchapter#1#2{第#1\putwordChapter{}\thischapterspace#2}

\gdef\Ynumbered{%
  \ifnum\secno=0
    第\the\chapno\putwordChapter%
  \else \ifnum\subsecno=0
    \the\chapno.\the\secno\putwordSection%
  \else \ifnum\subsubsecno=0
    \the\chapno.\the\secno.\the\subsecno\putwordSection%
  \else
    \the\chapno.\the\secno.\the\subsecno.\the\subsubsecno\putwordSection%
  \fi\fi\fi
}

\gdef\Yappendix{%
  \ifnum\secno=0
     \putwordAppendix@tie @char\the\appendixno{}%
  \else \ifnum\subsecno=0
     @char\the\appendixno.\the\secno\putwordSection%
  \else \ifnum\subsubsecno=0
     @char\the\appendixno.\the\secno.\the\subsecno\putwordSection%
  \else
     @char\the\appendixno.\the\secno.\the\subsecno.\the\subsubsecno\putwordSection%
  \fi\fi\fi
}

%%%%%%%%%%%%%%%%%%%%%%%%%%%%%%%%%
%  2. See 訳語に合わせて動詞の後置

\global\def\inforefzzz #1,#2,#3,#4**{\putwordInfo{}\putwordfile{} \file{\ignorespaces #3{}},  ノード\samp{\ignorespaces#1{}}\putwordSee{}}

\global\def\pxref#1{\xrefX[#1,,,,,,,]\putwordsee{}}
\global\def\xref#1{\xrefX[#1,,,,,,,]\putwordSee{}}

%%%%%%%%%%%%%%%%%%%%%%%%%%%%%%%%%
%  3. クロスレファレンスの表記

\global\def\xrefprintinmanual#1#2{\space \putwordin{} \cite{#1}}
\global\def\xrefprintrefinmanual#1#2{\cite{#2}の``#1''\putwordsection{}}
\global\def\xrefprintnodename#1#2#3{\def\temp{#1}\ifx\empty\temp\else#1\fi「#2」#3}

%%%%%%%%%%%%%%%%%%%%%%%%%%%%%%%%%
%  4. @dfn の表記

\global\def\doublebracket#1{『#1』}
\global\let\dfn=\doublebracket



%%%%%%%%%%%%%%%%%%%%%%%%%%%%%%%%%%%%%%%%%%%%%%%%%%%%%%%%%%%%%%%%
% for dvipdfmx
\gdef\usedvipdfmx{

  \ifnum 42146=\euc"A4A2 \special{pdf:tounicode EUC-UCS2}\else
  \special{pdf:tounicode 90ms-RKSJ-UCS2}\fi

  \global\pdfmakepagedesttrue
  \gdef\pdfdest name##1 xyz{%
    \iffinishedtitlepage\else
    \special{pdf:dest (name##1) [\@thispage /XYZ \@xpos \@ypos]}%
    \fi
  }

  \gdef\dopdfoutline##1##2##3##4{%
    \special{pdf:out [] ##2 << /Title (##1) /A << /S /GoTo /D (name##4) >> >> }%
  }

  \gdef\entryaddpdfoutline##1##2{%
    \global\cslet{orig##1}{##1}%
    \csdef{##1}####1####2####3####4{%
      \expandafter\empty\csname orig##1\endcsname{####1}{####2}{####3}{####4}%
      \dopdfoutline{####1}{##2}{####3}{####4}%
    }
  }

  \entryaddpdfoutline{numchapentry}{1}
  \entryaddpdfoutline{numsecentry}{2}
  \entryaddpdfoutline{numsubsecentry}{3}
  \entryaddpdfoutline{numsubsubsecentry}{4}

  \entryaddpdfoutline{unnchapentry}{1}
  \entryaddpdfoutline{unnsecentry}{2}
  \entryaddpdfoutline{unnsubsecentry}{3}
  \entryaddpdfoutline{unnsubsubsecentry}{4}

  \entryaddpdfoutline{appentry}{1}
  \let\appsecentry=\numsecentry
  \let\appsubsecentry=\numsubsecentry
  \let\appsubsubsecentry=\numsubsubsecentry

  \addcsbefore{contents}{%
    \special{pdf:docinfo <<
          /Title (\thistitle)
          %/Subject (subject)
          /Creator (texinfo.tex+dvipdfmx)
          %/Author (author)
          %/Producer (producer)
          %/Keywords (keywords)
        >>%
    }
  }
  \special{pdf:docview << /PageMode /UseOutlines >>}
}
%%%%%%%%%%%%%%%%%%%%%%%%%%%%%%%%%%%%%%%%%%%%%%%%%%%%%%%%%%%%%%%%
